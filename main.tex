% =========================
% Minimal Book Template
% =========================
\documentclass[11pt,openany]{book}

% ---------- Page / Font ----------
\usepackage[a4paper,margin=1in]{geometry}
\usepackage{setspace}
\setstretch{1.15}


% ---------- Links / Color ----------
\usepackage{xcolor}
\usepackage[hidelinks]{hyperref}

% ---------- TOC style (optional) ----------
\usepackage{tocloft}

\usepackage{braket}

% ---------- Boxes ----------
\usepackage[most]{tcolorbox}
\tcbset{
  enhanced,
  arc=10pt,                 % 圆角
  boxrule=0pt,              % 无边框线(更简约)
  left=10pt,right=10pt,
  top=10pt,bottom=10pt,
  before skip=10pt,
  after skip=10pt,
}

% 三种颜色的圆润方框(你可以改颜色/标题)
\newtcolorbox{InfoBox}[1][]{colback=blue!6, colframe=blue!40!black, title=Info, fonttitle=\bfseries, #1}
\newtcolorbox{NoteBox}[1][]{colback=green!6,colframe=green!40!black,title=Note, fonttitle=\bfseries, #1}
\newtcolorbox{WarnBox}[1][]{colback=orange!8,colframe=orange!55!black,title=Warning, fonttitle=\bfseries, #1}

% ---------- Code Listing Box ----------
\usepackage{listings}
\lstdefinestyle{clean}{
  basicstyle=\ttfamily\small,
  columns=fullflexible,
  breaklines=true,
  showstringspaces=false,
  tabsize=2,
  numberstyle=\tiny,
  numbers=left,
  xleftmargin=2.2em,
  frame=none,
  keywordstyle=\bfseries,
  commentstyle=\itshape,
}

% 代码专用方框:圆润 + 浅灰底 + 顶部标题
\newtcolorbox{CodeBox}[2][]{ % #2 是标题
  colback=black!3,
  colframe=black!20,
  arc=10pt,
  boxrule=0.6pt,
  title=\textbf{#2},
  fonttitle=\bfseries,
  listing only,
  listing options={style=clean},
  #1
}

% ---------- Title Info ----------
\title{\textbf{Quantum Communication and Networking}\\\large A Minimal \LaTeX\ Book Template}
\author{Your Name}
\date{\today}

\begin{document}

% =========================
% Cover Page
% =========================
\begin{titlepage}
  \centering
  \vspace*{2.2cm}
  {\Huge\bfseries Quantum Communication and Networking\par}
  \vspace{0.4cm}
  {\Large A NoteBook \par}
  \vspace{1.6cm}

  {\Large\textbf{Zhehao Yi}\par}
  \vspace{0.3cm}
  {\large \today\par}

  \vfill
  {\large \textit{Please have fun with quantum world}\par}
  \vspace*{1.5cm}
\end{titlepage}

% =========================
% Front Matter
% =========================
\frontmatter
\tableofcontents
\clearpage

% =========================
% Main Matter
% =========================
\mainmatter

\chapter{Quantum Information Science: An Introduction}

The three main aspects of quantum science and technology are:
\begin{itemize}
    \item Quantum Computing and Simulation (Optimization, Machine Learning, Material design, Drug discovery, etc.)
    \item Quantum Sensing (Magnetice field, Biomedical imaging, GPS-free navigation, etc.)
    \item Quantum Communication (Secure data encryption, Remote Q computing, Distributed quantum sensing, etc.)
\end{itemize}

Here we take a brief look at Quantum Computing:
\begin{NoteBox}{Quantum Computing}

Using qubits to perform computations that are infeasible for classical computers.

In traditional computers, bits are either 0 or 1. In quantum computers, qubits can be in superpositions of basis states, enabling potentially exponential speedups for certain problems. The basic states of a qubit can be represented as: $\ket{0}$ and $\ket{1}$.

So we can represent a qubit state as: $\ket{\psi} = \alpha\ket{0} + \beta\ket{1}$, where $\alpha$ and $\beta$ are complex numbers satisfying $|\alpha|^2 + |\beta|^2 = 1$.

In matrix form, we can write:
\begin{equation}
\ket{0} = \begin{pmatrix} 1 \\ 0 \end{pmatrix}, \quad \ket{1} = \begin{pmatrix} 0 \\ 1 \end{pmatrix}
\end{equation}

And there are various way to making qubits, such as 

\textit{SPIN QUBITS} (from spins of electrons or nuclei trapped in a solid substrate such as nitrogen vacancy cneters in diamond), 

\textit{PHOTONIC QUBITS} (from polarization or path of single photons), 

\textit{SUPERCONDUCTING QUBITS} (from Josephson junctions in superconducting circuits), 

\textit{ION TRAP QUBITS} (from internal states of trapped ions).

In quantum computing, we need error correction.
\end{NoteBox}

And now we look at Quantum Communication:
\begin{NoteBox}{Quantum Communication}

Using quantum states to transmit information securely over long distances.

Plateforms for quantum communication include satellite to ground or satellite to satellite links (long distance but low rate), ground to ground fiber links (potentially high rate with quantum repeaters).

Here we introduce some misconceptions:
\begin{itemize}
  \item \textbf{Misconception 1:} A qubit cannot have two values at the same time.
  \item \textbf{Misconception 2:} Measuring one particle cannot affect the other.
  \item \textbf{Misconception 3:} Quantum entanglement does not allow faster-than-light communication.
\end{itemize}

In quantum communication, we need network, repeaters, and protocols.
\end{NoteBox}

Next, we discuss quantum sensing:
\begin{NoteBox}{Quantum Sensing}

Using quantum systems to measure physical quantities with high precision and sensitivity.

Distributed quantum sensing can enhance the system's overall sensitivity by utilizing entanglement.

In quantum sensing, we need special materials, focus on doing well in economical, and also need networking.
\end{NoteBox}

Please remember:
\begin{WarnBox}
Please bear in mind that the uncertainty of quantum mechanics is fundamental, not due to lack of knowledge. This means that certain pairs of physical properties, like position and momentum, cannot be simultaneously known to arbitrary precision. This is a key principle that underpins the security of quantum communication protocols. Until you measure a quantum state, it exists in a superposition of all possible states. Measurement causes the state to collapse to one of the possible outcomes.
\end{WarnBox}


But we still have two quantum advantages:
\begin{NoteBox}{Quantum Advantages}

\textbf{Quantum Advantage 1:} We can finish certain known computational tasks faster, better, more precisely than classical methods (quantum computer).

\textbf{Quantum Advantage 2:} We can do certain task that is impossible for classical methods.
\end{NoteBox}



% chapter two
\chapter{The physical principle behind quantum networks}
What is your highest level understanding of quantum mechanics?

Don't worry, nothing will be scary. We will just take a brief look at the physical principle behind quantum networks.

\section{Quantum Information Science}
Let's first look at Classical Information Science:
\begin{NoteBox}{Classical Information Science}
Two types of classical information:
\begin{itemize}
  \item \textbf{Semantic information}: Meaningful information, such as text, images, audio, and video.
  \item \textbf{Technical information}: Is the set of symbols that are used.
\end{itemize}
\end{NoteBox}

What is a bit?
\begin{NoteBox}{Classical Bit}
A single bit can be in one of two states (ordinary bit), typically like a switch that can be either "on" (1) or "off" (0). And we call the condition of being in one of these two states as \textit{"STATE"}.
\end{NoteBox}

What is a memory cell?
\begin{NoteBox}{Memory Cell}
A memory cell is a box that containing two classical bits. So it can be in one of four states: 00, 01, 10, or 11. These probabilities mean different states. So we can use multiple bits to represent more complex information. Like 3 bits can represent 8 states (000 to 111), and so on. But in given time, a memory cell can only be in one of these states.
\end{NoteBox}

What is a quantum bit (qubit)?
\begin{NoteBox}{Quantum Bit (Qubit)}
A qubit is the quantum analog of a classical bit. Unlike a classical bit that can be either 0 or 1, a qubit can exist in a superposition and entanglement of both states simultaneously. This means that a qubit can be represented as:
\begin{equation} 
\ket{\psi} = \alpha\ket{0} + \beta\ket{1}, \quad \text{where } |\alpha|^2 + |\beta|^2 = 1
\end{equation}

Please bear in mind that you only can know the state of a qubit after you measure it. Before measurement, it exists in a superposition of both states.
\end{NoteBox}

So, compare to classical bits, in a "quantum memory cell" containing two qubits, the system can exist in a superposition of all four possible states (00, 01, 10, 11) simultaneously. This means that the quantum memory cell can represent and process multiple states at once, thanks to the principles of superposition and entanglement.

Here is a simple example about superposition and measurement:
\begin{NoteBox}{Superposition and Measurement Example}
Consider a qubit in the state:

If we prepare a qubit in 0 degree of rotation, and it spins clockwise, so after we measure it, we will get a qubit spinning clockwise with 100\% probability.

If we prepare a qubit in 90 degree of rotation, and it spins clockwise, so after we measure it, we will get a qubit spinning clockwise with 50\% probability, and counter-clockwise with 50\% probability.

If we prepare a qubit in 30 degree of rotation, and it spins clockwise, so after we measure it, we will get a qubit spinning clockwise with 80\% probability, and counter-clockwise with 20\% probability.
\end{NoteBox}

So what dose superposition mean?
\begin{NoteBox}{Superposition Meaning}
Evidently, superposition means that before measurement, the qubit is in a combination of both states (clockwise and counter-clockwise). The probabilities of measuring each state depend on the coefficients in the superposition. This is a fundamental principle of quantum mechanics that allows qubits to represent more information than classical bits.

And after measurement, the qubit collapses to one of the basis states (either clockwise or counter-clockwise) based on the probabilities determined by its superposition.

Please bear in mind that after measurement, the original superposition state is lost, and the qubit is now in a definite state. Which means you do a measurement again, you will get the same result with 100\% probability. Repeating measurements on the same qubit will yield the same outcome.
\end{NoteBox}

Let's play with two qubits entangled:
\begin{NoteBox}{Entangled Qubits Example}
We have two qubits, A and B, in the entangled state, which means, we have two small quantum memeory cells, one contains A and B but in 0 state, the other contains A and B but in 1 state. So the total state is:
\begin{equation}
\ket{\psi} = \frac{1}{\sqrt{2}}(\ket{00} + \ket{11})
\end{equation}
If we measure qubit A and find it in state 0, qubit B will instantaneously collapse to state 0 as well. 
Similarly, if we measure qubit A and find it in state 1, qubit B will collapse to state 1.
\end{NoteBox}

So you can't use this entanglement correlation to send information, it's only tell you the state of the other qubit after you measure one qubit.

\section{Encoding and transmitting quantum information}
Let's first look at the classical communication systems:
\begin{itemize}
  \item \textbf{Source}: Produces the message to be transmitted.
  \item \textbf{Transmitter}: Converts the message into a signal suitable for transmission.
  \item \textbf{Channel}: The medium through which the signal is transmitted (e.g., air, fiber optic cable). You may have noise in the channel.
  \item \textbf{Receiver}: Receives the transmitted signal and converts it back into a message.
  \item \textbf{Destination}: The final recipient of the message.
\end{itemize}

So what is a quantum communication network?
\begin{NoteBox}{Quantum Communication Network}
  A network of channels and nodes that transmits or shares quantum information. And quantum information is encoded in quantum states, such as qubits.
\end{NoteBox}

How we encode quantum information?
\begin{NoteBox}{Encoding Quantum Information}
  There are two types of qubits we can use to encode quantum information:
  \begin{itemize}
    \item \textbf{Stationary Qubits}: These qubits are most useful for storing quantum information. (e.g., electron spins states, the states of superconductor current, etc.)
    \item \textbf{Flying Qubits}: These qubits are most useful for transmitting quantum information over long distances. (e.g., photon polarization states, photon time-bin states, photo frequency, photo beam path, etc.)
  \end{itemize}
\end{NoteBox}

Encoding in photon polarization states:
\begin{NoteBox}{Photon Polarization Encoding}
  In photon polarization encoding, we use the polarization states of photons to represent qubit states. The two orthogonal polarization states, horizontal (H) and vertical (V), can be used as the basis states for a qubit.
  \begin{equation}
    \ket{0} \equiv \ket{H}, \quad \ket{1} \equiv \ket{V}
  \end{equation}

  So $V + H = D$ (Diagonal), $V - H = A$ (Anti-diagonal), and any polar direction can lead to any other polarization state so is $xH + yA$.

  But we need to pay attention that a simple phase change (e.g., from $H + V$ to $H - V$) can lead to a completely different state.
\end{NoteBox}

Here is a simple question for you:
\begin{InfoBox}{Question}
  Two photons are in the entangled state:
  \begin{equation}
    \ket{\psi} = \frac{1}{\sqrt{2}}(\ket{H_aV_b} + \ket{V_aH_b})
  \end{equation}

  If we measure photon a and find it in state $H$, what is the state of photon b after the measurement?

  Yes, the answer is $V$.
\end{InfoBox}

Another way is to encode information in photon by location (path) or in time of arrival.

How two transit a quantum state from one place to another?
\begin{NoteBox}{Transmitting Quantum States}
  First we need to know that there is no copy of an unknown quantum state. This is called the no-cloning theorem. So we can't use classical way to transmit quantum states.

  A quantum communication network must transmit the state of the physical system that encodes the qubit from one place to another, although it may do so by state teleportation.

  So we can use Joint Measurement tool, but the joint measurement can't tell us the state of each qubit before measurement. It only tell us the relation between the two qubits wheather they are the same or different.
\end{NoteBox}

Then we have entanglement swapping:
\begin{NoteBox}{Entanglement Swapping}
  First we have a "coin" $\alpha$ contains entangled state of A and B and a "coin" $\beta$ contains entangled state of C and D. The state equation of coin $\alpha$ and $\beta$ are
  \begin{equation}
    \begin{aligned}
      \mathrm{State}_{\alpha} &= \frac{1}{\sqrt{2}}(\ket{0_a1_b} + \ket{1_a0_b}) \\
      \mathrm{State}_{\beta} &= \frac{1}{\sqrt{2}}(\ket{0_c1_d} + \ket{1_c0_d})
    \end{aligned}
  \end{equation}

  Then, we take the joint measurement of B and C to get the outcome, and according to the outcome to distinguish the A and B.

  For example, if the outcome of joint measurement said B and C are the same, so there are two possibility either the B and C are the 0 or 1.

  If B and C are both 1, we can know the A is 0 and D is 0; If B and C are both 0, then the A is 1 and D is 1.
\end{NoteBox}


\section{Quantum State Teleportation}
Let's start with a simple example, Porf.Yi wants to send the quantum state of particle X to Porf.Zhang without sending the particle X itself. The state is $\mathrm{State_X} = x(0_x) + y(1_x)$. Because Prof.Yi cannot just measure the state and send the result to Prof.Zhang due to the no-cloning theorem, also, you only can get the limitation information from the measurement. So how can we do that? 

The answer is quantum entanglement. We first need to prepare an entangled pairs of particles A and B, the entangled state is $\mathrm{State_{AB}} = \frac{1}{\sqrt{2}}(0_a1_b + 1_a0_b)$. Then we send particle B to Prof.Zhang and send the particle A to Porf.Wang, and we do a joint measurement for particles A and X. And now, Prof.Wang can tell Prof.Zhang the outcome of the join measurement through classical channel, and then Prof.Zhang can do a corresponding operation on particle B through some controlled transformation device to convert the state of particle B to the original state of particle X. Now, Porf.Yi has successfully transmitted the quantum state of particle X to Prof.Zhang without sending the particle X itself, this called quantum state teleportation.

Quantum state teleportation can transfer the quantum state of a particle from one location to another without physically moving the particle itself, and nobady know the state of the particle during the process. But here we need to pay attention that we use a classical channel to send the joint measurement outcome, so the speed of quantum state teleportation cannot exceeed the speed of light.
\begin{NoteBox}
Quantum state teleportation is not instantaneous.
\end{NoteBox}

And the Particle X's original state is destroyed after the joint measurement, so there is no copy of the quantum state during the process, X no longer exist, he is sacrificed for the teleportation. And the state of particle X has been reproduce on particle B at Prof.Zhang's location.

Also, the quantum state teleportation can be used for the half entangled state.

After all, quantum state teleportation seems to be a synchronization problem. Because we both need classical channel and quantum entanglement channel to finish the teleportation. So, we come up with the idea of quantum memory.
\begin{NoteBox}{Quantum Memory}
  Quantum memory is a device that can absorbe a photon in a material medium in a way that its state is preserved.
\end{NoteBox}


\section{Bell State Measurements}
A quantum state is not a property of the photon; it is a description of the photon.

Born's rule:
\begin{NoteBox}{Born's Rule}
To find the probability for a photon to be observed passing throuigh a polarizer set for any given measurement scheme, project the photon;s polarization arrow onto the polarizer axis, then square the length of the projection.

So we will get the probability of the photon passing through the polarizer $a^2$. Then the complementary probability of the photon being absorbed by the polarizer is $b^2 = 1 - a^2$.
\end{NoteBox}

Here is another thing we need to pay attention:
\begin{NoteBox}
Single-particle experiments cannot rule out the possibility that nature follows inherent properties or inherent instruction tables.
\end{NoteBox}

Proof of the (classical) Bell inequality:
\begin{NoteBox}{Bell Inequality Proof}
Given these assumptions:
\begin{itemize}
  \item Realism: After the photons leave the common source, their inherent properties or instructions exist and don't change later.
  \item Causal Locality: Causal effects cannot travel faster than light.
  \item Free Will: Alice and Bob are able to make independent choices about what measurement each will make on each of their observed photons.
\end{itemize}
So the Bell inequality is, for any dataset $W, X, Y, Z$:
\begin{equation}
  \begin{aligned}
    Q &= W \times Y + W \times Z + X \times Y - X \times Z \\
    \mathrm{Avg}(Q) &= \mathrm{Avg}(W \times Y) + \mathrm{Avg}(W \times Z) + \mathrm{Avg}(X \times Y) - \mathrm{Avg}(X \times Z) \\
    \mathrm{Avg}(Q) &\leq 2
  \end{aligned}
\end{equation}
\end{NoteBox}

Quantum theory provides an explanation for correlations that works! The state description obeys local causality, but must be "global". A pairs of photons can be prepared in the entangled bell state: $\ket{\Phi^+} = \frac{1}{\sqrt{2}}(\ket{H_aH_b} + \ket{V_aV_b})$, and quantum theory predicts exactly the correlations observed in the bell test experiments. The experiments validate quantum theory and the fact that entangled states are an actual 'real'aspect of nature, which suggests that quantum states allow information processing and communication beyond what is possible with classical states.

\section{The quantum internet}




\begin{CodeBox}{Python Example}
def hello():
    print("Hello, LaTeX book!")
hello()
\end{CodeBox}

\begin{CodeBox}{C++ Example}
include <iostream>
int main(){
    std::cout << "Hello!" << std::endl;
    return 0;
}
\end{CodeBox}

\chapter{Second Chapter}


% =========================
% Back Matter
% =========================
\backmatter
\chapter*{Acknowledgments}


\end{document}
